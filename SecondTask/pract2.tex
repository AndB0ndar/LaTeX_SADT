%\graphicspath{{/home/arbon/Pictures/Screenshots/}} % path to graphics
\graphicspath{{~/Documents/SADT/SecondTask/}}
\chapter*{\LARGE{Цель практической работы}}
\addcontentsline{toc}{chapter}{Цель практической работы}
Данная практическая работа посвящена написанию базовых bash-скриптов.

По сути своей Bash-скриптов представляют из себя ни что иное как
последовательность команд командной строки, объединенных в один файл
для решения какой-либо задачи.

\chapter{Базовые Bash скрипты}

\paragraph{Cценарий, который выводит дату, время, список
зарегистрировавшихся пользователей, и uptime системы и сохраняет
эту информацию в файл}\mbox{}\par
Данный сценарий принемает параметром имя файла в который будет сохранятся
информация. Сам сценарий будет состоять из четырех команд:
\begin{itemize}
	\item \texttt{data} - выводит текущую дату и время;
	\item \texttt{cat} - выводит содержимое файла переданного ей параметром.
		В скрипте параметром передается файл \texttt{/etc/passwd},
		где храниться список всех пользователей;
	\item \texttt{cut} - обрезает строки, переданные ей на вход;
	\item \texttt{uptime} - возвращает информацию как долго система
		была запущена.
\end{itemize}

Код скрипта показан в листнинге \ref{lst:date}.
\lstinputlisting[language=Bash
	, caption=\leftline{Код скрипта}
	, label=lst:date]
	{./SecondTask/scripts/task1.sh}

\paragraph{Cценарий, который выводит содержимое любого каталога
или сообщение о том, что его не существует}\mbox{}\par
Для реализации данного скрипта используется управляющая конструкция
\texttt{if-then-else}. В условии используется команда \texttt{test},
ее сокращенная форма с квадратными скобками, и флаг \texttt{-d}
для проверки файла, является ли он директорией.

Код скрипта показан в листнинге \ref{lst:chdir}.
\lstinputlisting[language=Bash
	, caption=\leftline{Проверка на директорию}
	, label=lst:chdir]
	{./SecondTask/scripts/task2.sh}

\paragraph{Cценарий, который с помощью цикла прочитает файл и
выведет его содержимое}\mbox{}\par
Чтобы построчно прочитать файл использовался цикл \texttt{while} и в качестве
условию выступал результат команды \texttt{read line}, которая читает строки
из стандартного потока. Чтобы чтение происходило из файла, перенаправили
стандартный поток ввода в файл, переданный параметром.

Код скрипта показан в листнинге \ref{lst:cat}.
\lstinputlisting[language=Bash
	, caption=\leftline{Вывод содержимого файла}
	, label=lst:cat]
	{./SecondTask/scripts/task3.sh}

\paragraph{Cценарий, который с помощью цикла выведет список
файлов и директорий из текущего каталога, укажет, что есть файл, а
что директория}\mbox{}\par
\lstinputlisting[language=Bash
	, caption=\leftline{Проверка на директорию}
	, label=lst:chdir]
	{./SecondTask/scripts/task4.sh}

\paragraph{Cценарий, который подсчитает объем диска, занимаемого
директорией}\mbox{}\par
\lstinputlisting[language=Bash
	, caption=\leftline{Проверка на директорию}
	, label=lst:chdir]
	{./SecondTask/scripts/task5.sh}

\paragraph{Cценарий, который выведет список всех исполняемых
файлов в директории, для которых у текущего пользователя есть права
на исполнение}\mbox{}\par
\lstinputlisting[language=Bash
	, caption=\leftline{Проверка на директорию}
	, label=lst:chdir]
	{./SecondTask/scripts/task6.sh}

