\chapter*{Ответы на вопросы}
\addcontentsline{toc}{chapter}{Ответы на вопросы}

\begin{enumerate}
	\item \textbf{Какие существуют типы систем контроля версий?
		Приведите примеры к каждому типу.}\\
		Системы контроля версий бывают локальными, централизованными или
		распределёнными. Локальная система хранит файлы на одном устройстве,
		централизованная использует общий сервер, а распределённая --- общее
		облачное хранилище и локальные устройства участников команды.
		В локальной системе удобно работать с большими проектами,
		но сложно взаимодействовать с удалённой командой.
		Одним из примеров \textit{локальных} систем является система
		контроля версий RCS.
		К \textit{централизованным} относятся CVS, Subversion, Perforce.
		К \textit{распределённому типу} систем контроля версий относятся
		Mercurial, Bazaar, Darcs и Git.
	\item \textbf{Что такое репозиторий Git?}\\
		Репозиторий Git --- это каталог, в котором Git отслеживает изменения.
		На компьютере может быть любое количество репозиториев,
		каждый из которых хранится в собственном каталоге.
	\item \textbf{Для чего нужен .gitignore?}\\
		Файл .gitignore предназначен для исключения из индексации Git файлов
		и каталогов проекта. Размещаться этот файл может в любом каталоге
		проекта и количество этих файлов в разных каталогах не ограничено.
		Зона действия файла распространяется от каталога в которой он лежит
		и на все вложенные файлы и каталоги.
	\item \textbf{Что делает команда git status?}\\
		Она выводит информацию обо всех изменениях, внесенных в дерево
		директорий проекта по сравнению с последним коммитом рабочей ветки;
		отдельно выводятся внесенные в индекс и неиндексированные файлы.
	\item \textbf{Что сделает команда "git branch" без какого-либо параметра?}\\
		Эта команда выведет список существующих веток.
	\item \textbf{Что означает статус файла untracked в выводе команды
		git status?}\\
		Статус Untracked означает, что Git видит файл, которого не было
		в предыдущем снимке состояния (коммите); Git не станет добавлять
		его в коммиты, пока его явно об этом не попросить.
	\item \textbf{Как сделать ветку с названием my\_branch?}\\
		Для создания ветки используется команда \texttt{git branch <имя ветки>}.
	\item \textbf{Что такое GitHub?}\\
		GitHub — крупнейший веб-сервис для хостинга IT-проектов
		и их совместной разработки. Веб-сервис основан на системе
		контроля версий Git
\end{enumerate}

\chapter*{Выводы}
\addcontentsline{toc}{chapter}{Выводы}
В данной практической работы были получены основные навыки по работе с Git,
такие как: настройка git; создание репозитория; индексация изменений и коммиты.

Научились управлять репозиторием: создавать ssh ключ, связывать локальный и
удаленный репозиторий, создавать ветки и перемещаться между ними.

Также узнали о ветвлении: создание форка репозитория, клонирование репозитория
слияние веток, устранение конфликтов слияния.
