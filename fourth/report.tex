\graphicspath{{./fourth/img/}}

\section*{\LARGE Введение}
\addcontentsline{toc}{section}{Введение}
\textbf{Цель работы:} получить навыки настройки вычислительной
инфраструктуры при помощи системы конфигурационного управления
Ansible
\textbf{Задание:}
Написать роль для запуска сервера nginx, написать playbook для
применения роли, провести тестовый запуск playbook’а, в случае успешного
прохождения теста, применить playbook к серверам.\par
Необходимо добавить переменную, содержащую ФИО, номер группы и
номер варианта. Данная переменная должна выводиться в шаблонный файл
nginx.\par
Установка пакета выполняется при помощи модуля APT, используемого
для установки nginx в базовой роли.\par
Добавьте в playbook task по установке пакета согласно варианту:
3. wget.

\clearpage

\section{Индивидуальные задания}
Хостовая машина является основной с операционной системой Ubuntu. В 
качестве управляемой машины будет применялась виртуальная машина с 
установленной системой Ubuntu. Доступ к управляемой машине 
реализовывался с помощью сетевого моста, активированного в Visual Box и 
SSH протокола.\par
Для начала произведем настройку при помощи файла
\textit{/etc/network/interfaces}. Открыв его в редакторе vim, увидем набор
параметров для двух интерфейсов lo и wlo1. Второй
интерфейс необходимо настроить по следующему подобию.

\begin{image}
	\includegrph{Screenshot from 2023-05-22 18-06-32}
	\caption{Конфигурация интерфейса}
\end{image}

Чтобы узнать адрес той сети, в которой расположена
хостовая машина в Linux есть команда: \texttt{hostname -I}.\par
Ещё одним важным шагом является установка ssh-сервера, который
позволит удалённо подключаться к машине. Установить его можно при
помощи команды:

\begin{verbatim}
apt install ssh
\end{verbatim}

Теперь можем написать роль для запуска сервера nginx.\par
Сперва вернёмся в рабочую папку со всеми файлами для
Ansible и создадим там директорию \textit{roles}. После чего перейдём в эту
директорию и инициируем роль стандартной структуры при помощи команды:

\begin{verbatim}
ansible-galaxy init nginx
\end{verbatim}

Перейдём к созданию роли. Заполним соответствующие файлы данными
из секций playbook’а и соответствующие директории ранее созданными
файлами.\par
Это значит, что в директории tasks файл main.yml должен быть
заполнен данными из секции tasks. В директории vars --- из секции vars. В
директории handlers --- из секции handlers. Также файлы из директорий
files и templates должны быть перемещены в директории files и
templates в папке роли.\par

\begin{lstlisting}[language=xml
	, caption=\leftline{Playbook для запуска роли}
	]
---
- name: Install and config Nginx via Role
  hosts: webservers
  become: yes

  roles:
    - nginx
\end{lstlisting}

\begin{lstlisting}[language=xml
	, caption=\leftline{Сценарии task, используемые в роли}
	]
# tasks file for nginx
---
- name: Install Nginx
  apt:
    name=nginx state=present update_cache=yes
  when:
    ansible_os_family == "Debian"
  notify:
    - Nginx Systemd

- name: Delete default HTML files
  shell: /bin/rm -rf /usr/share/nginx/html/*.html
  args:
    warn: false

- name: Replace config file
  vars:
    nginx_user: user
    worker_processes: 2
    worker_connections: 256
  template:
    src: templates/nginx.conf.j2
    dest: /etc/nginx/nginx.conf
    mode: 0644
  register: result
  failed_when: result.failed == true
  notify: Reload Nginx

- name: Copy index file
  copy: src=files/index.html dest={{ html_dir }} mode=0644
  notify: Reload Nginx

- name: Generate dynamic HTML from template
  template:
    src=templates/who.html.j2 dest={{ html_dir }}/who.html owner=root mode=0644
  notify: Reload Nginx

- name: Install wget
  apt:
    name=wget state=present update_cache=yes
\end{lstlisting}

\begin{lstlisting}[language=xml
	, caption=\leftline{Переменные, используемые в роли}
	]
---
# vars file for nginx
  html_dir: /usr/share/nginx/html
  greeting: Бондарь Андрей Ренатович, ИКБО-06-21, 3
\end{lstlisting}

\begin{lstlisting}[language=xml
	, caption=\leftline{Обработчики (handlers), используемые в роли}
	]
---
# handlers file for nginx
- name: Nginx Systemd
  systemd:
    name: nginx
    enabled: yes
    state: started
- name: Reload Nginx
  systemd: name=nginx state=reloaded
\end{lstlisting}

\begin{lstlisting}[language=xml
	, caption=\leftline{Шаблон страницы, используемый в роли}
	]
"Server {{ ansible_hostname }} ( ip {{ansible_default_ipv4.address }} )
	greets you: {{ greeting | default("Hello") }}!"
\end{lstlisting}


\begin{lstlisting}[language=xml
	, caption=\leftline{Шаблон конфигурационного файла, используемый в роли}
	]
# Пользователь, из-под которого будет запущен nginx
user {{ nginx_user }};
# Количество рабочих процессов, которые будет задействовать nginx
worker_processes {{ worker_processes }};
error_log /var/log/nginx/error.log warn;
pid /var/run/nginx.pid;
events {
 # Число соединений, которое может поддерживать каждый пр оцесс
	worker_connections {{ worker_connections }};
}
http {
	include /etc/nginx/mime.types;
	default_type application/octet-stream;
	keepalive_timeout 60;
	gzip on;
	server {
	access_log off;
		location / {
			root /usr/share/nginx/html;
			try_files \$uri \$uri/ \$uri.html /index.html;
		}
	}
}
\end{lstlisting}

После написания роли его можно выполнить при помощи
следующей команды: \texttt{ansible-playbook -i hosts <имя файла роли>}.

\begin{image}
	\includegrph{Screenshot from 2023-05-22 18-14-14}
	\caption{Запуск ansible}
\end{image}

Результат запуска сервера можно проверить утилитой \texttt{curl}:

\begin{image}
	\includegrph{Screenshot from 2023-05-22 18-30-17.png}
	\caption{Запуск ansible}
\end{image}

\clearpage

\section*{\LARGE Выводы}
\addcontentsline{toc}{section}{Выводы}
В данной работе были приобретены навыки работы с системой 
управления конфигурациями Ansible. В том числе написание собственных
playbook'ов и ролей.\par
Более того, были получены 
компетенции в области работы с оболочкой bash и протоколом SSH.

